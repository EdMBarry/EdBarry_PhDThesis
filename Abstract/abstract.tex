% %TO PRODUCE A STAND-ALONE PDF OF YOUR ABSTRACT, un-comment this header and the \end{document} at the end of the file.
% %
%\documentclass[12pt, a4paper]{memoir}
%
%\usepackage{mathptmx}
%\input{../style.tex}
%
%\begin{document}
%
%\begin{center}
	%\textbf{\large Your title goes here}
	%
	%\textbf{Abstract}
	%
	%Your Name, The University of Queensland, 20??
%\end{center}

%WRITE YOUR ABSTRACT HERE.
Purple phototrophic bacteria (PPB) are known to exist in a variety of environments and have been demonstrated as a potentially effective biotechnology in several industries such as wastewater treatment, cosmetics production, and energy production. Although PPB processes are gaining relevance in the aforementioned industries, physical limitations exist which are providing resistance to the widespread adoption of PPB based biotechnology. This thesis has aimed to define the most important limitations and has sought to define their physical behaviour in a mathematical framework. The first limitation was that there was no PPB process model describing the behaviour of PPB in the presence of ammonia, organic matter, and other nutrients. The mixed population models of the International Water Association (IWA) family of models describe species whose metabolism differ from that of PPB but they provide a good framework for process modelling and control. As such, a process model for a mixed-culture PPB system was developed based on five key processes: photoheterophic growth, photoautotrophic growth, chemoheterotrophic growth, hydrolysis/fermentation, and decay. This model was developed as a set of ordinary differential equations varying in time, and lumped in space. The sharp attenuation of the radiative field in a PPB system means that not all the biomass experiences the same radiative field in a reactive domain. This means that the model required an extension to be able to describe the system as it varies in both space and time. A CFD modelling framework was therefore developed in OpenFOAM to describe the spatial variations and interactions between biomass growth, the flow field, and the radiative field. It was found that lumped modelling approaches and distributed parameter approaches differed in results based on different reactor domains and behaviours. The deviation from lumped parameter behaviour was greater for a cylindrical stirred reactor than for a flat plate reactor, which means that spatial considerations are necessary in the design phase of a photobioreactor. Finally, technical questions around the harvesting of PPB biomass present significant engineering challenges and are a limitation to the adoption of this biotechnology. The model was thus extended to include biofilm formation of a mixed PPB system in the presence of a spatially varying radiative field. A volume-of-fluid approach was used as a basis for this model, where particulate matter was segregated to the phase volume fraction, and the growth of the three particulate species (phototrophic bacteria, biodegradable particulate matter, and inert particulates) were included as source terms to the phase volume fraction equations, which in turn influenced the momentum equations, resulting in biofilm growth in space.

The combination of all of these methods provide a working PPB modelling framework which can form the basis to better understand PPB (and phototrophic) systems within a process modelling context, and with further experimental analysis, can provide a foundation for a photobioreactor design and optimisation project. 

%Start this section on a new page [this template will automatically handle this].
%The abstract should outline the main approach and findings of the thesis and must be between 300 and 800 words.

%\end{document}