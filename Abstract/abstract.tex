% %TO PRODUCE A STAND-ALONE PDF OF YOUR ABSTRACT, un-comment this header and the \end{document} at the end of the file.
% %
%\documentclass[12pt, a4paper]{memoir}
%
%\usepackage{mathptmx}
%% *************** Document style definitions ***************

% ******************************************************************
% This file defines the document design.
% Usually it is not necessary to edit this file, but you can use it to change aspects of the design if you want.
% ******************************************************************

%------------------------------------------------------------------------------%
%----------------------------LOAD PACKAGES-------------------------------------%
%------------------------------------------------------------------------------%

% Feel free to alter/add to these packages as you need.
% ******************* Load packages *******************
%Miscellaneous.
\usepackage{cite}									%Allows abbreviated numerical citations.
\usepackage[figuresright]{rotating}	%Allows large tables to be rotated to landscape.
\usepackage{pdfpages}							%Allows you to include full-page pdfs.
\usepackage{wrapfig}							%Lets you wrap text around figures.
%Maths stuff.
\usepackage{bm} 									%Bolded maths characters.
\usepackage{upgreek}							%Upright Greek characters.
\usepackage{dsfont}								%Double-struck fonts.
%\usepackage{simplewick}						%For typesetting Wick contractions.
\usepackage{mathtools}						%Can be used to fine-tune the maths presentation.	
%Text packages.
\usepackage{framed}								%For boxed text.
\usepackage{microtype}						%pdfLaTeX will fix your kerning.
\usepackage{marvosym}							%Include symbols (like the Euro symbol, etc.).
%Figures.
	\usepackage{color}							%Nice for scalable pdf graphics using InkScape.
	\usepackage{transparent}				%Nice for scalable pdf graphics using InkScape.
\usepackage{placeins}							%Lets you put in a \FloatBarrier to stop figures floating past this command.
%Lists.
\usepackage{mdframed,mdwlist} 		%Use these for nice lists (less white space).

%------------------------------------------------------------------------------%
%---------------------MACROS-----THE-BLACK-------------------------------------%
%------------------------------------------------------------------------------%

%Define a bunch of macros that implement Latin abbreviations.
%COMMENT OUT OR DELETE IF UNDESIRED.
\newcommand{\via}{\textit{via}} %Italicised via.
\newcommand{\ie}{\textit{i.e.}} %Literally.
\newcommand{\eg}{\textit{e.g.}} %For example.
\newcommand{\etc}{\textit{etc.}} %So on...
\newcommand{\vv}{\textit{vice versa}} %And the other way around.
\newcommand{\viz}{\textit{viz}.} %Resulting in.
\newcommand{\cf}{\textit{cf}.} %See, or 'consistent with'.
\newcommand{\apr}{\textit{a priori}} %Before the fact.
\newcommand{\apo}{\textit{a posteriori}} %After the fact.
\newcommand{\vivo}{\textit{in vivo}} %In the flesh.
\newcommand{\situ}{\textit{in situ}} %On location.
\newcommand{\silico}{\textit{in silico}} %Simulation.
\newcommand{\vitro}{\textit{in vitro}} %In glass.
\newcommand{\vs}{\textit{versus}} %James \vs{} Pete.
\newcommand{\ala}{\textit{\`{a} la}} %In the manner of...
\newcommand{\apriori}{\textit{a priori}} %Before hand.
\newcommand{\etal}{\textit{et al.}} %And others, with correct punctuation.
\newcommand{\naive}{na\"\i{}ve} %Queen Amidala is young and \naive{}.

% *************** End of document style definition ***************
%
%\begin{document}
%
%\begin{center}
	%\textbf{\large Your title goes here}
	%
	%\textbf{Abstract}
	%
	%Your Name, The University of Queensland, 20??
%\end{center}

%WRITE YOUR ABSTRACT HERE.
Purple phototrophic bacteria (PPB) use infrared light to generate microbial energy, with electron, carbon and nitrogen supplied chemically. They have broad applicability for resource recovery from wastes and wastewater, as they have potential value as microbial product and very high growth yields on substrate. Although PPB processes are gaining relevance in the aforementioned industries, reactor design, particularly effective consideration of light in chemical kinetics and hydraulics is a critical issue. This thesis aims to assess key limitations and define their physical behaviour mathematically. The first limitation was that there was no mixed culture PPB process model describing the behaviour of PPB in the presence of ammonia, organic matter and other nutrients. The mixed population models of the International Water Association (IWA) provide a good framework for process modelling and control. As such, a process model for a mixed-culture PPB system was developed based on five key processes: photoheterotrophic growth, photoautotrophic growth, chemoheterotrophic growth, hydrolysis/fermentation, and decay. This model was developed as a set of ordinary differential equations varying in time and lumped in space (as for the IWA models). The second limitation is that the radiative field has not been considered. This can vary spatially and temporally. The radiative field in a PPB system attenuates sharply, which influences the growth domains. A CFD modelling framework was therefore developed in OpenFOAM to describe the spatial variations and interactions between biomass growth, the flow field and the radiative field. It was found that lumped modelling approaches and distributed parameter approaches differed in results based on different reactor domains and behaviours. The deviation from lumped parameter behaviour was greater for a cylindrical stirred reactor than for a flat plate reactor. Therefore, spatial considerations are necessary in the design phase of a photobioreactor. Finally, biofilms are a critical aspect in PPB growth, and a coupled biofilm radiative transfer model has not been previously considered. The model was thus extended to include biofilm formation of a mixed PPB system in the presence of a spatially varying radiative field. A volume-of-fluid approach was used as a basis for this model, considering three particulate species (phototrophic bacteria, biodegradable particulate matter, and inert particulates). The radiative-solid-liquid coupling overall was found to be critical in operation, with traditional lumped parameter approaches to design and analysis not well suited to the emerging challenges of mixed culture photo-bio systems. These aspects should be further considered through model based analysis and experimental validation for future system design.

%\end{document}