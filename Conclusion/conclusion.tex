\chapter[Conclusions and outlook]{Conclusions and outlook}
\label{chap:conclusion}

\section{Conclusions}
Purple phototrophic bacteria, and more broadly, photosynthetic and phototrophic microorganisms have presented as possible biotechnological solutions to waste treatment and the production of highly valued products such as pigments and agricultural feeds. Their widespread adoption has however been limited due several biochemical and physical limitations, including the delivery of radiation, the mass transfer limitations, mixing requirements, and solid/liquid separation limitations. 

A useful approach to minimise the effects physical and biological limitations is to model the process. Through mechanistic modelling, a deeper understanding of the underlying physics of the system, as well as a platform for design and optimisation can be achieved. This thesis has sought to provide a semi-mechanistic modelling framework in order to address the aforementioned limitations associated with photobioreactors. A process model was developed to describe the metabolism of phototrophic bacteria in anaerobic conditions within a process engineering context. This was required as a basis to understand the limitations of light delivery under an array of operating conditions and designs. As the radiative field within a photobioreactor sharply attenuates within the domain, it was required to extend the biochemical model couple the growth and reactions with spatially varying radiative and hydrodynamic fields. In response to the difficulties in harvesting phototrophic biomass, a biofilm model was developed to show how its growth is coupled to a radiative field. Phototrophic growth in biofilms is desirable as energy-intensive membrane systems can be avoided for biomass/liquid separation. The computational framework has been developed using free and open source software (FOSS) which means that it can be freely downloaded, improved and redistributed. This means that outside of the learning curve associated with Linux-based systems, OpenFOAM and Python, the barriers for contributing to and extending this body of work are minimal.

\subsection{Description of a biochemical process model for phototrophic growth}




\textbf{Paragraph 2: Overarching goal of the thesis}
This work was done in order to present a modelling framework for the growth of purple phototrophic bacteria. The goal of the framework was to describe the behaviours of PPB in domestic or medium strength wastewaters when subject to substrate limitations, including limitations due to insufficient irradiance. From this mathematical framework, it is expected that model validation and calibration will lead to an industrial tool for process understanding, design and optimisation. 

\textbf{Paragraph 3: Small paragraph into the met objectives of the thesis.}




\textbf{Objective 1: }







\textbf{Objective 2: }





\textbf{Objective 3: }


\section{Significance of the research outcomes}





\section{Outlook and future work}