\chapter[Conclusions and outlook]{Conclusions and outlook}
\label{chap:conclusion}

\section{Conclusions}
Purple phototrophic bacteria, and more broadly, photosynthetic and phototrophic microorganisms have presented as possible biotechnological solutions to waste treatment and the production of highly valued products such as pigments and agricultural feeds. Their widespread adoption has however been limited due several biochemical and physical limitations, including the delivery of radiation, the mass transfer limitations, mixing requirements, and solid/liquid separation limitations. 

A useful approach to minimise the effects physical and biological limitations is to model the process. Through mechanistic modelling, a deeper understanding of the underlying physics of the system, as well as a platform for design and optimisation can be achieved. This thesis has sought to provide a semi-mechanistic modelling framework in order to address the aforementioned limitations associated with photobioreactors. A process model was developed to describe the metabolism of phototrophic bacteria in anaerobic conditions within a process engineering context. This was required as a basis to understand the limitations of light delivery under an array of operating conditions and designs. As the radiative field within a photobioreactor sharply attenuates within the domain, it was required to extend the biochemical model couple the growth and reactions with spatially varying radiative and hydrodynamic fields. In response to the difficulties in harvesting phototrophic biomass, a biofilm model was developed to show how its growth is coupled to a radiative field. Phototrophic growth in biofilms is desirable as energy-intensive membrane systems can be avoided for biomass/liquid separation. The computational framework has been developed using free and open source software (FOSS) which means that it can be freely downloaded, improved and redistributed. This means that outside of the learning curve associated with Linux-based systems, OpenFOAM and Python, the barriers for contributing to and extending this body of work are minimal.

\subsection{Description of a biochemical process model for phototrophic growth}

A model was developed for the growth of a mixed culture purple phototrophic bacteria system. The model development and implementation was focused on domestic wastewater treatment but could have been extended to growth on other substrates. It was found through a series of batch tests that photoheterotrophic growth was predominant, and photoautotrophic and chemoheterotrophic growth were both considerably slower (roughly one order of magnitude when comparing maximum specific uptake rates).  The rate of decay was higher than that for activated sludge processes and hydrolysis was slower, meaning that biodegradable particulates do not degrade at short solid retention times. 

The mechanisms were explored in greater detail with the aid of batch simulations. The first set of simulations highlighted the requirement of additional readily biodegradable SCOD in order to consume all inorganic nitrogen and phosphorus. For medium strength domestic wastewater, 400 gCOD m\textsuperscript{-3} are required for full inorganic nitrogen and phosphorus assimilation. The second batch simulations looked at the daily cycling between phototrophic and chemotrophic growth for \textit{e.g.} an outdoor photobioreactor depending on direct solar irradiation. Two different specific uptake rates for chemoheterotrophy were used (k\textsubscript{M,ch} = 0.07 and 0.7 d\textsuperscript{-1}). In the case of the higher chemoheterotrophic activity, an increased biomass growth during dark cycles led to acetate accumulation for consumption during light cycles. There was also a net production of hydrogen which meant that substrate was available for photoautotrophic growth during light cycles as well. A final batch simulation was done to analyse biomass behaviour under photoautotrophic activity. With sufficient H\textsubscript{2} as substrate, the determined specific photoautotrophic uptake rate k\textsubscript{M,ic} was compared with those found in literature (\num{0.0034} and \num{0.025} molC gCOD\textsuperscript{-1}d\textsuperscript{-1}). This was to show that while photoautotrophic uptake was insignificant when compared to photoheterotrophic uptake, a change of waste stream leading to a switch to a predominant inorganic electron donor could change the system's behaviour. By using the uptake rates obtained in literature, carbon dioxide fixation could account for more than 30\% of inorganic N and P removal.

The model was then tested in continuous conditions to see how seasonal changes could affect the behaviour of a PPB system for domestic wastewater treatment. The response to seasonal changes was generally rapid. In addition to the simulated seasonal changes, there were two distinct changes in operating conditions, where periods I and III included no additional SCOD, and period II included the addition of SCOD to align with inorganic nitrogen and phosphorus concentrations (a control dosing mechanism). During all periods, average SCOD removal didn't change (87\%) with the remaining SCOD almost entirely consisting of non-biodegradable components. With no additional SCOD, ammonium and phosphate removal efficiencies were 47\% and 59\% respectively. With additional SCOD, the phosphorus removal efficiency achieved 93\%. The near complete removal of phosphorus had an impact on nitrogen removal efficiency, which averaged 74\%. This study also depicted biomass fractionation over continuous operation. From period I to period II, the proportion of solids as PPB biomass increased from 44\% to 53\%, due to the additional SCOD leading to a higher sustained growth by photoheterotrophic metabolism. Biodegradable particulates (X\textsubscript{S}) were accumulated within the reactor; a direct consequence of the low hydraulic retention time imposed. 



\subsection{Inclusion of hydrodynamics and radiative transfer}



%\textbf{Paragraph 2: Overarching goal of the thesis}
%This work was done in order to present a modelling framework for the growth of purple phototrophic bacteria. The goal of the framework was to describe the behaviours of PPB in domestic or medium strength wastewaters when subject to substrate limitations, including limitations due to insufficient irradiance. From this mathematical framework, it is expected that model validation and calibration will lead to an industrial tool for process understanding, design and optimisation. 



\section{Significance of the research outcomes}





\section{Outlook and future work}
