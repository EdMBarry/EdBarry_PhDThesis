\chapter[Conclusions and outlook]{Conclusions and outlook}
\label{chap:conclusion}

\section{Conclusions}
Purple phototrophic bacteria, and more broadly, photosynthetic and phototrophic microorganisms have presented as possible biotechnological solutions to waste treatment and the production of highly valued products such as pigments and agricultural feeds. Their widespread adoption has however been limited due several biochemical and physical limitations, including the delivery of radiation, the mass transfer limitations, mixing requirements, and solid/liquid separation limitations. 
\skippingparagraph
A useful approach to minimise the effects physical and biological limitations is to model the process. Through mechanistic modelling, a deeper understanding of the underlying physics of the system, as well as a platform for design and optimisation can be achieved. This thesis has sought to provide a semi-mechanistic modelling framework in order to address the aforementioned limitations associated with photobioreactors. A process model was developed to describe the metabolism of phototrophic bacteria in anaerobic conditions within a process engineering context. This was required as a basis to understand the limitations of light delivery under an array of operating conditions and designs. As the radiative field within a photobioreactor sharply attenuates within the domain, it was required to extend the biochemical model couple the growth and reactions with spatially varying radiative and hydrodynamic fields. In response to the difficulties in harvesting phototrophic biomass, a biofilm model was developed to show how its growth is coupled to a radiative field. Phototrophic growth in biofilms is desirable as energy-intensive membrane systems can be avoided for biomass/liquid separation. The computational framework has been developed using free and open source software (FOSS) which means that it can be freely downloaded, improved and redistributed. This means that outside of the learning curve associated with Linux-based systems, OpenFOAM and Python, the barriers for contributing to and extending this body of work are minimal.

\subsection{Description of a biochemical process model for phototrophic growth}

A model was developed for the growth of a mixed culture purple phototrophic bacteria system. The model development and implementation was focused on domestic wastewater treatment but could have been extended to growth on other substrates. It was found through a series of batch tests that photoheterotrophic growth was predominant, and photoautotrophic and chemoheterotrophic growth were both considerably slower (roughly one order of magnitude when comparing maximum specific uptake rates).  The rate of decay was higher than that for activated sludge processes and hydrolysis was slower, meaning that biodegradable particulates do not degrade at short solid retention times. 
\skippingparagraph
The mechanisms were explored in greater detail with the aid of batch simulations. The first set of simulations highlighted the requirement of additional readily biodegradable SCOD in order to consume all inorganic nitrogen and phosphorus. For medium strength domestic wastewater, 400 gCOD m\textsuperscript{-3} are required for full inorganic nitrogen and phosphorus assimilation. The second batch simulations looked at the daily cycling between phototrophic and chemotrophic growth for \textit{e.g.} an outdoor photobioreactor depending on direct solar irradiation. Two different specific uptake rates for chemoheterotrophy were used (k\textsubscript{M,ch} = 0.07 and 0.7 d\textsuperscript{-1}). In the case of the higher chemoheterotrophic activity, an increased biomass growth during dark cycles led to acetate accumulation for consumption during light cycles. There was also a net production of hydrogen which meant that substrate was available for photoautotrophic growth during light cycles as well. A final batch simulation was done to analyse biomass behaviour under photoautotrophic activity. With sufficient H\textsubscript{2} as substrate, the determined specific photoautotrophic uptake rate k\textsubscript{M,ic} was compared with those found in literature (\num{0.0034} and \num{0.025} molC gCOD\textsuperscript{-1}d\textsuperscript{-1}). This was to show that while photoautotrophic uptake was insignificant when compared to photoheterotrophic uptake, a change of waste stream leading to a switch to a predominant inorganic electron donor could change the system's behaviour. By using the uptake rates obtained in literature, carbon dioxide fixation could account for more than 30\% of inorganic N and P removal.
\skippingparagraph
The model was then tested in continuous conditions to see how seasonal changes could affect the behaviour of a PPB system for domestic wastewater treatment. The response to seasonal changes was generally rapid. In addition to the simulated seasonal changes, there were two distinct changes in operating conditions, where periods I and III included no additional SCOD, and period II included the addition of SCOD to align with inorganic nitrogen and phosphorus concentrations (a control dosing mechanism). During all periods, average SCOD removal didn't change (87\%) with the remaining SCOD almost entirely consisting of non-biodegradable components. With no additional SCOD, ammonium and phosphate removal efficiencies were 47\% and 59\% respectively. With additional SCOD, the phosphorus removal efficiency achieved 93\%. The near complete removal of phosphorus had an impact on nitrogen removal efficiency, which averaged 74\%. This study also depicted biomass fractionation over continuous operation. From period I to period II, the proportion of solids as PPB biomass increased from 44\% to 53\%, due to the additional SCOD leading to a higher sustained growth by photoheterotrophic metabolism. Biodegradable particulates (X\textsubscript{S}) were accumulated within the reactor; a direct consequence of the low hydraulic retention time imposed. 



\subsection{Inclusion of hydrodynamics and radiative transfer}
This study involved the development and comparison of four different mathematical models and explored how different resolutions in the physics of the system gave different results. The model development used the distributed parameter biokinetic framework described in Chapter 2 and extended that base to include radiative transfer and fluid flows in for both single and two-phased systems. 
\skippingparagraph
The models adopted different layers of complexity which had previously been applied in the literature: a lumped parameter system using a uniform incident irradiance, a lumped system using a uniform irradiance measured halfway into the domain of the reactor, a system of ordinary differential equations using a dynamic input irradiance as determined by a hydrodynamic field, and a full Eulerian CFD solution. 
\skippingparagraph
This study confirmed the hypothesis that the dependency of the system on the radiative field would mean that a lumped parameter approach would not be an appropriate modelling simplification for photobioreactors. The spatial varying nature of the radiative field meant that hydrodynamics and the coupling of the radiative field to the biochemical equations played an important role in describing the system. Generally, the lumped parameter approaches over-predicted growth rates. The results from the particle-radiation dynamics approach approached that of the full Eulerian CFD solution. This can serve as an appropriate simplification when computational infrastructure is lacking for a full CFD solution. 

\subsection{Development of a photobiofilm model}
This study aimed to address the practical and economic concerns related to the separation of phototrophic biomass from liquid streams. In this work a continuum biofilm model was developed and its growth was tracked in two and three dimensional domains with varying radiative source configurations. 
\skippingparagraph
This model was developed using the volume of fluid (VOF) method in OpenFOAM, where both biofilm and liquid were treated as a single fluid with differing characteristic parameters. All particulate species were confined to the biofilm phase fraction, and their growths were incorporated as source terms into the balance equation for volume fraction. The volume fraction state variable was coupled to the momentum equations and growth could ensue.

%\textbf{Paragraph 2: Overarching goal of the thesis}
%This work was done in order to present a modelling framework for the growth of purple phototrophic bacteria. The goal of the framework was to describe the behaviours of PPB in domestic or medium strength wastewaters when subject to substrate limitations, including limitations due to insufficient irradiance. From this mathematical framework, it is expected that model validation and calibration will lead to an industrial tool for process understanding, design and optimisation. 


\section{Significance of the research outcomes}
The development of the biokinetic model was done in such a way that it can be used as a module in the IWA family of models, leading to a broader applicability in plant-wide wastewater treatment simulations and control systems. The model can also be adapted to different applications of purple phototrophic bacteria, such as the production of hydrogen, or growth on non-waste substrates for the production of high value products such as astaxanthin and other pigments. Outside of model applicability and parameter values, the structure of the model provides insight into the metabolism of PPB at a high level of abstraction than metabolic pathway models. This allows the reader to develop an intuition into the behaviour of phototrophic biomass under different operating conditions. 
\skippingparagraph
The extension of the biokinetic model into a distributed parameter framework which describes the hydrodynamics, radiation and the biokinetics allows for a greater understanding of how these physical processes interact. As such, this modelling framework can be used as an optimisation tool where photobioreactor designs seek to minimise the process bottlenecks presented by mixing and light delivery limitations. The development of the \texttt{photoBio} radiative transfer libraries could be implemented in domains outside of the direct use case of photobioreactors. They could be used to model algal or cyanobacterial blooms for prevention purposes. They could additionally be used to describe plant growth in both outdoor and indoor illuminated settings. The inactivation of pathogens, or advanced oxidation processes through ultra-violet radiation could also be modelled using these libraries. The use of CFD in ultra-violet photocatalytic and photolysis processes has been used, but the open nature of the software means that more resources can be attributed to simulation jobs without the associated licensing fees. Further afield, the behaviour of photo-voltaic cells could be modelled using the software developed during this body of work. Although a wide array of applications could benefit from this work, there are some fields for which components of this model would not be appropriate. Most notably, the modelling of medical imaging processes is usually done using Monte Carlo photon flux models, which are both more computationally expensive, and more accurate than the finite volume discrete ordinates method.
\skippingparagraph
This general CFD framework for phototrophic and photosynthetic organisms has been implemented using an open source CFD platform. All extensions to the model must also include the same open source license. This increases accessibility of the model and allows for a wider community to propose improvements and bug fixes, potentially leading to a more robust piece of software. 


\textbf{Biofilm conclusions in their broader context}



\section{Outlook and future work}
